\documentclass{article}

\usepackage{graphicx}
\usepackage{float}
\usepackage{verbatim}
\usepackage{amssymb}
\usepackage{amsmath}
\usepackage{hyperref}
\usepackage{authblk}
\usepackage{algpseudocode}
\usepackage{algorithm}

\setlength\itemsep{1em}
\hyphenation{E-the-re-um at-tes-ta-tion}

\title{Exact Approaches for the\\Attestation Aggregation and Packing Problem}
\author{Satalia \& Sigma Prime}

\begin{document}

\maketitle{}

\begin{abstract}
We explore two exact approaches for solving the Attestation Aggregation and
Packing Problem (AAPP, \cite{Satalia22a}). Both approaches are
optimality-preserving.

The first approach (which we refer to as the \emph{MIP approach}) involves a
first stage to deal with the \emph{aggregation} part of the problem, and a
second stage, based on a a mixed integer programming (MIP) formulation, to deal
with the \emph{packing} part of the problem. The second approach (which we
refer to as the \emph{decomposition approach}) decomposes the problem into a
main problem and many sub-problems. The main problem can be solved efficiently
using dynamic programming, the sub-problems are are smaller instances of a
special case of the AAPP, which can be solved optimally with any exact
approach.

The purpose of the decomposition approach is to help with the potential
scalability issues of the MIP approach on larger instances.
\end{abstract}

\tableofcontents

\newpage

\section{Introduction}

\newcommand{\attestations}{\ensuremath{A}}
\newcommand{\solution}{\ensuremath{S}}
\newcommand{\attester}{\ensuremath{v}}
\newcommand{\attesters}[1]{\ensuremath{V_{#1}}}
\newcommand{\allattesters}{\ensuremath{V}}
\newcommand{\epoch}[1]{\ensuremath{e(#1)}}
\newcommand{\attestation}{\ensuremath{a}}
\newcommand{\data}[1]{\ensuremath{d_{#1}}}
\newcommand{\Data}{\ensuremath{D}}
\newcommand{\epochs}{\ensuremath{E}}
\newcommand{\Reward}[1]{\ensuremath{R\left(#1\right)}}
\newcommand{\reward}[2]{\ensuremath{r\left(#1, #2\right)}}

The present document outlines two exact approaches for solving the Attestation
Aggregation and Packing Problem (AAPP) defined in \cite{Satalia22a}. For
convenience, we will not include the formal definition of the problem here.
Instead, we will assume familiarity with \cite{Satalia22a} and commit to use
the same notation. Where a symbol is used in different contexts, e.g., the
symbol $e$ for epochs and set elements, the meaning will be clear from the
context.

While the main use case for the proposed exact approaches is to find optimal
solutions to the AAPP, we will take scalability into account for practical
reasons.

\section{The MIP approach} \label{sec:mip}

The MIP approach is based on the idea of re-framing the AAPP as a combination
of an \emph{aggregation problem} and a \emph{packing problem}. This is similar
in spirit to the approach taken by Sigma Prime's Lighthouse\footnote{Ethereum
consensus client.}. Like Lighthouse, the MIP approach solves these two problems
in successive stages. Unlike Lighthouse, the MIP approach is complete and is
guaranteed to (eventually) find an optimal solution.

 
\subsection{Observations}

Let $\attestations^{\dagger} \subseteq 2^{\attestations}$ be the set of
sets of attestations satisfying the aggregation conditions, i.e.,
%
\begin{equation}
  \attestations^{\dagger} = \{ B~|~\attesters{a} \cap \attesters{b} = \emptyset
  \land \data{a} = \data{b},~\forall a, b \in B,~B \in 2^{\attestations} \}
\end{equation}
%
Note that any set $S \subseteq \attestations^{\dagger}, |S| \leq N$ is a
feasible solution of the AAPP, and that each element of
$\attestations^{\dagger}$ represents a valid\footnote{In the sense of the
conditions for aggregation.} aggregated attestation. The re-framing of the AAPP
behind the MIP approach is based on the following two Observations
%
\begin{enumerate}
  \item[\textbf{O1}] Because every attestation, regardless of the size of its
  attester set, takes up exactly one of $N$ slots within a block, including an
  attestation $a \in \attestations^{\dagger}$ with attester set $\attesters{a}$
  in a solution is non-worse than including an attestation $b \in
  \attestations^{\dagger}$ with attester set $\attesters{b} \subseteq
  \attesters{a}$,
  \item[\textbf{O2}] An attester $\attester \in \allattesters$ appearing in a
  solution $S$ contributes only once to the reward of that solution, regardless
  of how many times it appears in the attester sets of attestations included in
  $S$.
\end{enumerate}
%
The consequence of \emph{O1} is that attestations that are \emph{maximal} with
respect to attester coverage are the best (or at least non-worse) candidates
for inclusion in a solution of the AAPP. The set of such attestations is the
set
%
\begin{align}
  \attestations^{\star} = \{ B~|~\not\exists C \in \attestations^{\dagger},
  \attesters{B} \subset \attesters{C}, B \in \attestations^{\dagger} \}
\end{align}
%
In other words, all attestations in $\attestations^{\star}$ are
\emph{non-dominated} with respect to the coverage of their attester set, i.e.,
they are non-worse than any other attestations in $\attestations^{\dagger}$ at
covering their particular attester set. Note that these attestations are also
\emph{maximally aggregated}, i.e., they cannot be aggregated with any other
attestation in $\attestations^{\dagger}$. If they were, they wouldn't be
maximal with respect to attester coverage\footnote{If they were, it would be
possible to find an attestation with the same attestation data and disjoint set
of validators to aggregate them with, which would lead to a higher attester
coverage, which directly contraddicts the definition.}. Note that, according to
the above definition, $\attestations^{\star}$ could contain aggregated
attestations that are equivalent with respect to attester coverage. This is not
a limitation for our approach.

This has direct implications for the search of an \textbf{optimal} solution
because, given a set of attestations $\attestations{}$, there is at least some
optimal solution \solution{} to the AAPP which is a subset of
$\attestations^{\star}$.  It has also implications for the search of a
\textbf{good quality} (but not necessarily optimal) solutions since, due to the
fixed capacity $N$, attestations not in $\attestations^{\star}$ are less likely
to be part of a good-quality solution.

Considering $2^{\attestations^{\star}}$, as opposed to
$2^{\attestations^{\dagger}}$, as the search space has two main advantages
%
\begin{itemize}
  \item first, the former represents a smaller search space, i.e.,
  $|2^{\attestations^{\star}}| \leq |2^{\attestations^{\dagger}}|$, and
  \item second, because every element of ${\attestations^{\star}}$ already
  satisfies all the aggregation constraints, we can safely ignore them going
  forward.
\end{itemize}
%
Computing $\attestations^{\star}$ is what we refer to as the \emph{aggregation
problem}, which is the focus of the first stage of our MIP approach.

The consequence of \emph{O2} is that, once the aggregation problem has been
solved, the AAPP reduces to choosing $N$ aggregated attestations from
$\attestations^{\star}$ so as to maximise the reward to be gained by including
the votes of the attesters they cover. This is a \emph{weighted maximum
coverage problem}, which needs to be solved exactly in order for our approach
to guarantee optimality. This is what we refer to as the \emph{packing
problem}.

\subsection{Aggregation problem} \label{sec:clique}

In this section we outline a method to compute $\attestations^{\star}$ using a
graph representation of the set of attestations \attestations{}.

Let $G = (V,E)$ be an undirected graph with vertices $V$ and edges $E$, where
$V = \attestations$, and 
%
\begin{equation} 
  E = \{ (a, b)~|~\{a, b\} \in \attestations^{\dagger} \}
\end{equation}
%
i.e., each vertex represents one of the attestations in \attestations{}, and each
edge encodes the compatibility for aggregation of the attestations it connects. 

A \emph{clique} of $G$ is a set $C \subseteq V$ such that each pair of vertices
in the clique is connected by an edge, i.e., $(a, b) \in E, \forall a, b \in C,
a \neq b$. Note that, due to our encoding of $G$, a clique represents an
aggregate attestation from attestations in \attestations{}, and the set of all
such attestations is $\attestations^{\dagger}$.  A \emph{maximal clique} is a
clique that cannot be further extended with another vertex, and thus
corresponds to a maximally aggregated attestation from \attestations{}, and the
set of all such attestations is $\tilde{\attestations} \subseteq
\attestations^{\dagger}$. Note that while the elements of
$\tilde{\attestations}$ are maximally aggregated attestations, they are not
necessarily non-dominated, and therefore $\tilde{\attestations} \supseteq
\attestations^{\star}$. For instance, consider the two hypothetical maximally
aggregated attestations $a, b \in \tilde{\attestations}$, their respective
attester sets could be 
%
\begin{align}
  \attesters{a} = \{ v_1, v_2, v_3 \} \\
  \attesters{b} = \{ v_1, v_2, v_3, v_4 \}.
\end{align}
%
Obviously $a$ is not maximal with respect to attester coverage, since choosing
$b$ is non-worse than choosing $a$ in terms of covering $\{ v_1, v_2, v_3 \}$.
In other words, $a \notin \attestations^{\star}$ but $b \in
\attestations^{\star}$.

In the following, we present a method to compute $\attestations^{\star}$ by
first computing $\tilde{\attestations}$ by enumerating all the maximal cliques,
and then eliminating all the aggregated attestations that are dominated with
respect to attester coverage, i.e., those whose attester sets are proper
subsets of of the attester set of some other attestation in
$\tilde{\attestations}$. Again, this is not a strict requirement for the
approach, but is likely to make the search space more compact.

\paragraph{Note.}For the purpose of solving the AAPP with the MIP approach, it
would be sufficient to choose $N$ attestations from $\attestations^{\dagger}$.
Computing $\attestations^{\star}$ is a way to reduce the search space of the
packing problem while preserving the optimality of the approach.
\vspace{2em}

\noindent
Most of the existing algorithm for enumerating maximal cliques are variants of
the Bron-Kerbosch \cite{Bron73} algorithm. This is a recursive algorithm that
maintains three sets of vertices $R$, $P$, and $X$ respectively modeling the
maximal clique being built, the vertices that could be included in the clique,
and the vertices that won't be included in the clique. The essence of the
Bron-Kerbosch algorithm is summarised in the pseudo-code below, where we denote
by $N(v)$ the set of vertices connected by some edge to a vertex $v \in V$ of a
given graph $G = (V,E)$.

\begin{algorithm}
\caption{Original Bron-Kerbosch}
\begin{algorithmic}
\Procedure{BronKerbosch}{$R, P, X$}
\If{$P \cup X = \emptyset$}
  \State report $R$ as maximal clique
\EndIf
\For{$v \in P$}
  \State \Call{BronKerbosch}{$R \cup \{v\}$, $P \cap N(v)$, $X \cap N(v)$}
  \State $P \gets P \setminus \{v\}$
  \State $X \gets X \cup \{v\}$
\EndFor
\EndProcedure
\end{algorithmic}
\end{algorithm}

While the original Bron-Kerbosch algorithm would correctly produce the set of
maximal cliques in a graph, the algorithm has been improved since its first
inception in 1973. Two main developments are the use of \emph{pivoting}
\cite{Koch01, Cazals05} and the use of \emph{vertex ordering}
\cite{Eppstein11}. The above techniques can be applied to the general maximal
clique enumeration problem and are therefore relevant for us. 

\paragraph{Pivoting.}
%
The idea of pivoting is based on the observation that, given a vertex $p \in P
\cup X$, any maximal clique must contain $p$ or one of its non-neighbors,
i.e., $P \setminus N(p)$ (otherwise, the clique could be extended by adding $p$
to it), therefore only non-neighbors of $p$ need to be explored while
extending $R$ (otherwise a non-maximal clique is being explored). A strategy
for choosing $p$ that has proven to be effective both theoretically and
experimentally is to choose
%
\begin{equation}
  p = \underset{p \in P \cup X}{\arg\min} |P \setminus N(p)|.
\end{equation}
%
This variant would roughly translate to the following pseudo-code.

\begin{algorithm}[H]
\caption{Bron-Kerbosch with pivoting}
\begin{algorithmic}
\Procedure{BronKerbosch}{$R, P, X$}
\If{$P \cup X = \emptyset$}
  \State report $R$ as maximal clique
\EndIf
\State $p \gets \underset{p \in P \cup X}{\arg\min} |P \setminus N(p)|$
\For{$v \in P \setminus N(p)$}
  \State \Call{BronKerbosch}{$R \cup \{v\}$, $P \cap N(v)$, $X \cap N(v)$}
  \State $P \gets P \setminus \{v\}$
  \State $X \gets X \cup \{v\}$
\EndFor
\EndProcedure
\end{algorithmic}
\end{algorithm}

\paragraph{Vertex ordering.}
%
The vertex ordering variant is based on the idea that, by choosing the ordering
in which the recursive calls are made, i.e., by choosing the ordering in which
vertices in $P$ are explored, one can reduce the size of the recursive search
tree. As for the possible orderings, using a \emph{degeneracy ordering} has
been associated with better worst-case run time guarantees. In order to obtain
a degeneracy ordering, one approach is to start with an empty ordering and
iteratively remove a minimum degree vertex from the graph and append it to the
ordering until the graph is empty. This is done in the outermost level of the
recursion, while the innermost levels still use a pivoting strategy.

\begin{algorithm}[H]
\caption{Bron-Kerbosch with ordering}
\begin{algorithmic}
\Procedure{BronKerboschOrdering}{$G = (V, E)$}
\State $v_1, v_2, \dots, v_n \gets$ \Call{ComputeOrdering}{G}
\For{$i \in \{1, \dots, n\}$}
  \State $P \gets \{ v_j | j > i\} \cap N(v_i)$
  \State $R \gets \{ v_i \}$
  \State $X \gets \{ v_j | j < i\} \cap N(v_i)$
  \State \Call{BronKerbosch}{$R, P, X$}
\EndFor
\EndProcedure
\end{algorithmic}
\end{algorithm}

Aside from the above improvements to the original Bron-Kerbosch algorithm, the
fact that $V = \attestations{}$ allows us to consider two additional
optimisations. First, some of the attestations in \attestations{} will
expectedly be unaggregated, and will therefore correspond to vertices connected
with all the vertices for attestations that do not include them, which could be
many. This can increase the cost of enumerating all cliques, and can be handled
as a pre-processing and a post-processing step, discussed below under
\emph{unaggregated attestations}. Second, the aggregation conditions state that
two vertices cannot be connected if their attestation data differs.  This means
that $G = (V,E)$ as defined above could contain many disconnected sub-graphs
partitioned by attestation data and, as a consequence, the set of maximal
cliques will be partitioned in the same way. Solving the maximal clique
enumeration for each attestation data independently reduces the size of the
graphs to be processed, which we cover \emph{attestation data partitioning}
below.

\paragraph{Unaggregated attestations.}
%
We want to reduce the complexity of enumerating all the cliques for graphs
arising in the context of the AAPP. Such graphs are of the form $G = (V =
\attestations, E)$. In particular, we're interested in dealing with
unaggregated attestations, which are compatible with many cliques and can
therefore contribute to generating a large set of maximal cliques. 

Let us denote by $\attestations^{1} = \{ a~|~a \in \attestations,~
|\attesters{a}| = 1 \}$ the set of unaggregated attestations in
$\attestations$. Our strategy for dealing with these is to remove them from $G$
altogether, enumerate all the maximal cliques for $G^{\prime} = (\attestations
\setminus \attestations^{1}, E^{\prime})$, where $E^{\prime} = \{ (a, b)~|~(a,
b) \in E, a, b \in \attestations \setminus \attestations^{1} \}$, and then add
all $a \in \attestations^{1}$ back in all the compatible cliques. We
denote this set as $\tilde{\attestations{}^{\prime}}$.

Note that in general $\tilde{\attestations{}^{\prime}} \subseteq
\tilde{\attestations{}}$. This is not a limitation, as it can be shown that
$\tilde{\attestations{}^{\prime}} \supseteq \attestations^{\star}$ holds. In
other words, the additional cliques that would have been generated by
considering $\attestations^{1}$, i.e., $\tilde{\attestations{}} \setminus
\tilde{\attestations{}^{\prime}}$ are necessarily dominated by the ones in
$\tilde{\attestations{}^{\prime}}$. 

\paragraph{(Potential) attestation data partitioning.}
%
As mentioned above, the graph $G = (V, E)$ is partitioned by attestation data.
This suggests that the maximal clique enumeration can be decomposed, which may
have a positive impact on the performance. Let $\attestations_{d} = \{ a~|~a
\in \attestations,~\data{a} = d \},~\forall d \in \Data$. We can then generate
all the aggregated attestations that are maximal with respect to attester
coverage $\attestations_{d}^{\star}$ for $\attestations_{d}$, by defining
$G_{d} = (\attestations_{d}, E)$ where $E$ is defined in the obvious way. And
following the approach described above. Then, the set of all maximal cliques
for \attestations{} can be then defined as
%
\begin{equation}
  \attestations^{\star} = \underset{d \in \Data}{\bigcup}
  \attestations_{d}^{\star}.
\end{equation}
%
The intuition is that calculating $\attestations_{d}^{\star},~\forall d \in
\Data$ is more efficient than computing $\attestations^{\star}$ for
$\attestations{}$ as a whole\footnote{This needs to be proven experimentally.}.


\paragraph{Other approaches.}

The presented algorithms are well-established, and tend to achieve good performance
on generic graphs, and have the advantage of simplicity, which can be a
desirable property for the type of use case discussed here.
However more sophisticated algorithms exist.  Most of these target graphs with
particular properties. On these graphs, these algorithms tend to provide better
worst-case complexity guarantees.  Some others address generic graphs, and can
provide better guarantees than the algorithms based on Bron-Kerbosch. We take
stock of these algorithms, which can be explored in a successive phase if
necessary.

\paragraph{Note.} whichever technique is chosen to enumerate the set
$\attestations^{\star}$, its efficiency is crucial to the viability of both
approaches proposed here. In the following, we assume that
$\attestations^{\star}$ has been found in a way or another, and is available as
an input to the packing stage.

\subsection{Packing problem} \label{sec:coverage}

As we have mentioned above, the set of all aggregated attestations that are
maximal with respect to attester coverage $\attestations^{\star}$ is an
optimality-preserving subset of the search space of the AAPP. 
%
\newcommand{\weight}[1]{\ensuremath{w(#1)}}
%
Based on this premise, the AAPP can be easily formulated as a \emph{weighted
maximum coverage problem}. 

The weighted maximum coverage problem involves choosing at most $k$ from a set
of sets $S = \{ S_1, \dots, S_n \}$, where each element $e \in \bigcup_{i \in
\{1, \dots, n\}} S_i$ is associated with a weight $\weight{e} \in
\mathbb{N}^{\geq 0}$. The goal of the problem is to maximise the sum of the
weights of the elements that are part of at least one of the $k$ sets that are
chosen to be part of the solution. Like its non-weighted variant, the weighted
maximum coverage problem is unfortunately NP-hard. It be formulated as a mixed
integer program (MIP) as follows
%
\begin{align}
  \mathbf{maximise}~  & \underset{e \in E}{\sum}w(e_j) \cdot y_j \\ \label{eq:cov}
  s.~t.~              & \sum x_i \leq k \\ \nonumber
                      & \underset{e_j \in S_i}{\sum}x_i \geq y_j \\ \nonumber
                      & y_j \in \{0, 1\}  \\ \nonumber
                      & x_i \in \{0, 1\} \nonumber
\end{align}
%
where $S = \{ S_1, \dots, S_n \}$ is the set of sets that can be included in a
solution and $E = \{ e_1, \dots, e_m \} = \bigcup_{S_i \in S} S_i$ is the set
of elements in any of the sets. The variables $x_i \in \{1, \dots, n\}$ and
$y_j \in \{1, \dots, m\}$ encode, respectively, the decision to choose a set
$S_i$ in to be part of the solution, and the fact that element $e_j$ is covered
by the solution.

\paragraph{Mapping.}Under the premise that $\attestations^{\star}$ is
available, the mapping of the AAPP to this problem is trivial. Let $S = \{
\attesters{a}~|~a \in \attestations^{\star} \}$, and $k = N$, where $N$ has the
meaning defined in \cite{Satalia22a}. A solution for the weighted maximum
coverage problem defined this way is a solution to the original AAPP.
Moreover, an optimal solution to this problem is an optimal solution of the
original AAPP.

This formulation can be directly plugged into a MIP solver to find an optimal
packing of $N$ maximally aggregated attestations. 

\paragraph{Other exact approaches.}Of course, the weighted maximum coverage
problem can be solved optimally by any other exact approach. Such an approach,
e.g., constraint programming (CP), would likely benefit from a different
modeling. In this section we focus on MIP because of the wide availability of
these kinds of solvers.

\paragraph{Greedy algorithm.} A greedy algorithm exists and is indeed the
algorithm currently employed by the Lighthouse client developed by Sigma Prime.
This greedy algorithm has a guaranteed approximation ration of $1 - 1/e
\approx0.623$ of the optimum, which is not suitable for the purpose of this
work, but may be of interest for the follow-up implementation of an efficient
algorithm. Note that in Lighthouse, the packing problem is defined on a set of
attestations that have been pre-aggregated heuristically, which differs from
the approach presented here.

\section{Decomposition approach}

Solving the AAPP using a MIP solver may represent a suitable approach, unless
the problem is too large. For these scenarios, we propose to use a
decomposition approach.

The main idea behind this approach is to reduce the complexity of solving the
AAPP by separating the parts that can be handled efficiently with \emph{dynamic
programming} from the ones that are strictly \emph{NP-hard}. In particular,
this approach is based on the observation that, when choosing which
attestations to include in a solution, the contribution of attestations with
different attestation data is additive, i.e.,
%
\begin{equation}
  \data{a} \neq \data{b} \Rightarrow Reward{\{ a, b \}} = \Reward{\{a\}} +
  \Reward{\{b\}},~ \forall a, b \in \attestations.
\end{equation}
%
This suggests that, by partitioning an instance of the AAPP by attestation
data, we could re-frame it as the problem of choosing the number $q_d \in
\mathbb{N}^{\geq 0}$ of sets of attestations to include from each
$\attestations_{d}$, where $d \in \Data$. This is reminiscent of the Knapsack
Problem \cite{Martello87}, which can be solved efficiently with dynamic
programming.

In order for a solution to the AAPP to be optimal, one needs to choose the
\emph{best} $q_d$ attestations for $d \in \Data$, i.e., the ones that will
collectively provide maximal reward. This includes of course both attestations
in $\attestations_{d}$ but also aggregated attestations
$\attestations_{d}^{\dagger}$ from attestations in $\attestations_{d}$.

This is, once again a \emph{weighted maximum coverage problem} (albeit
expectedly a much smaller one), and thus NP-hard. The value of this
decomposition is that it allows us to treat the full AAPP as a combination of a
dynamic programming problem (which we will refer to as \textbf{main problem})
and many small NP-hard problems (which we will refer to as
\textbf{sub-problems}), as opposed to one large NP-hard problem. This is
typically a much better outlook.

In the rest of this section, we will characterise both the main problem and the
sub-problems. Because the former depends on the latter, we will discuss the
sub-problems first.

\subsection{Sub-problems}

Let $d \in \Data$ be a unique attestation data in the input. We denote by
$\attestations_{d}$ the set of all attestations for $d$, and $k \in
\mathbb{N}^{\geq 0}$ a positive integer. We now consider the AAPP problem
defined by $A = \attestations_{d}$ and $N = k$, and denote by $g(d, k)$ any
optimal solution for such AAPP. For a given $d \in \Data$ and $k \in
\mathbb{N}^{\geq 0}$, $g(d, k)$ can be found using the MIP approach presented
in Section~\ref{sec:mip}, or any equivalent exact approach. 

\paragraph{Note} For some $k \in \mathbb{N}^{\geq 0}$ it is possible that
$|g(d, k)| < k$. This represents a sort of \emph{fixpoint} for $g$, and
reflects two possible scenarios, either
%
\begin{itemize}
  \item there are fewer than $k$ sets of attestations from
  $2^{\attestations_{d}}$ to choose from, or
  \item $\Reward{g(d, k)} = \Reward{g(d, k-1)}$, i.e., adding more sets of
  attestations doesn't increase the value of $g(d, k)$, i.e., $\attesters{g(d,
  k-1)} \supseteq \{ v~|~v\in \attesters{\attestations_{d}},~
  \reward{\epoch{d}}{\attester} > 0 \}$.
\end{itemize}
%

\subsection{Main problem}

Now that we have defined the sub-problems and that we can refer to their
optimal solutions, we have all the ingredients to define the main problem as a
dynamic programming problem.

Let $\Data = \{ d_1, \dots, d_n \}$ be a set of $n \in \mathbb{N}^{\geq 0}$
distinct attestation data, and let $m \in \mathbb{N}^{\geq 0}$ be a
non-negative integer. We observe that, under the premise that $g(d_i, k)$ for
some $i \in \{1, \dots, n \}$ and some $k \in \{ 0, \dots, m \}$ is the optimal
(most rewarding) set of $k$ aggregated attestations
$2^{\attestations_{d}^{\dagger}}$, the AAPP reduces to finding an appropriate
value $q_d \in \{ 0, \dots, m \}$ for every $d$ such that
%
\begin{equation}
  \underset{d \in \Data}{\sum} q_d \leq m, \label{eq:capm}
\end{equation}
%
in other words, $q_{d_1}, \dots, q_{d_n}$ uniquely identify a packing $S$ of at
most $m$ aggregated attestations, where for each $d \in \Data$ we're choosing
the $q_d$ best attestations to include. The actual packing can then be derived
as
%
\begin{equation}
   S = \underset{d \in \Data}{\bigcup} g(d, q_d).
\end{equation}

To find the optimal solution, we proceed with a dynamic programming approach.
We denote by $f(k, m)$ the \emph{optimal} packing of $m$ aggregated
attestations $\bigcup_{i \in 1, \dots, k} 2^{\attestations_{d_i}^{\dagger}}$
from the attestation sets of the first $k$ attestation data.  As customary, we
first identify a base case, and then define the non-base cases recursively.

\paragraph{Base case.}
% 
Note that
%
\begin{align}
  f(0, m) = \emptyset, \;\; \forall m \in \mathbb{N}^{\geq 0}\\
  \Reward{f(0,m)} = 0
\end{align}
%
i.e., the only solution for the main problem considering no attestation data at
all\footnote{It is possible to define the base case as $f(1, m)$ however its
definition is already encompassed by the definition for the non-base cases, so
starting from 0 is more succinct.} is the empty set, which yields a reward of
0, no matter the available capacity. 

\paragraph{Non-base cases.}
%
Based on this, we can build our non-base case as follows
%
\begin{align}
  f(k, m) = \underset{q \in \{ 0, \dots, \min\left(m, |g(k, m)|\right)
  \}}{\arg\max}~\Reward{ f(k-1, m-q) \cup g(d_k, q) }
\end{align}
%
i.e., the best packing $f(k, m)$ of at most $m$ sets of aggregatable
attestations considering only the first $k$ attestation data $\{ d_1, \dots,
d_k \} \subseteq \Data$ is the set that maximises the total reward that can be
obtained by choosing the best $m-q$ aggregated attestations from
either of the $f(k-1, m-q)$, and the best $q$ aggregated attestations
from $g(d_k, q)$. 

Note that the domain $\{0, \dots, \min\left(m, |g(k, m)|\right) \}$ of $q$
takes into account the fact that it is possible that $g(k, m) < m$ for some
values of $m$, and therefore adding more sets than available in $g(k, m)$ has
no meaning. 

The optimality of this approach is guaranteed by the fact that $f(k-1, m-q)$ is
the optimal packing for attestation data $\{ d_1, \dots, d_{k-1} \}$ and a
maximum capacity of $m-q$ and, when extending our options to attestation sets
for $d_k$, we consider all values of $q$ from $0$ (which corresponds to not
including any attestation set for $d_k$) to $m$ (which corresponds to choosing
all $m$ sets from the best $m$ attestation sets for $d_k$). Note that at any
time, only one value of $q_i,~i \in \{1, \dots, n\}$ is chosen. The fact that
attestation sets for different attestation data do not overlap in terms of
their contribution to a solution, means that the reward of a given packing is
additive, and can be computed as
%
\begin{equation}
  \Reward{f(k, m)} = \underset{q \in \{0, \dots, m\}}{\max} \Reward{f(k-1,
  m-q)} + \Reward{g(d_i, q)}.
\end{equation}

It follows that $f(n, N)$, i.e., the optimal packing considering all
attestation data and a maximum capacity of $N$ (where $N$ has the same meaning
as in \cite{Satalia22a}), corresponds to the optimal solution of the original
AAPP.

Note that the property $|f(k, m)| \leq m$ in Equation~\ref{eq:capm} implies
that it is possible that $|f(k, m)| < m$ for some $m \in \mathbb{N}^{\geq 0}$.
Like for $g(\cdot, \cdot)$, this suggests a sensible stopping condition for the
search for $f(k, m)$. This stopping condition can be used to avoid exploring
$f(k, m+1)$ if $|f(k, m)| = |f(k, m-1)|$.

\paragraph{Improvements.} Given that finding each $g(d, k)$ is NP-hard, it
makes sense to help the MIP approach solve the sub-problems by injecting as
much information as we can from the main problem. One information that we have
available when requesting a $g(d, k)$ is the reward of $g(d, k-1)$, i.e.,
$Reward(g(d, k-1))$. We know that adding more aggregated attestations
to $g(d, k)$ must at least achieve the same reward, plus the reward to be
obtained by greedily including the single most rewarding aggregated
attestation not already in $g(d, k-1)$. Let denote such attestation by $g^{+}_{d,
k-1}$. We can now inject the following additional constraint in the MIP model
of Equation~\ref{eq:cov}
%
\begin{equation}
  \underset{e \in E}{\sum}w(e_j) \cdot y_j \geq \Reward{g(d,k-1)} + \Reward{
  g^{+}_{d,k-1}}.  
\end{equation}
%
Our expectation is that such a constraint could help prune branches of
the MIP search tree that cannot reach a better-than-greedy quality.


\bibliographystyle{plain}
\bibliography{approaches}

\end{document}
